\PassOptionsToPackage{unicode=true}{hyperref} % options for packages loaded elsewhere
\PassOptionsToPackage{hyphens}{url}
%
\documentclass[]{article}
\usepackage{lmodern}
\usepackage{amssymb,amsmath}
\usepackage{ifxetex,ifluatex}
\usepackage{fixltx2e} % provides \textsubscript
\ifnum 0\ifxetex 1\fi\ifluatex 1\fi=0 % if pdftex
  \usepackage[T1]{fontenc}
  \usepackage[utf8]{inputenc}
  \usepackage{textcomp} % provides euro and other symbols
\else % if luatex or xelatex
  \usepackage{unicode-math}
  \defaultfontfeatures{Ligatures=TeX,Scale=MatchLowercase}
\fi
% use upquote if available, for straight quotes in verbatim environments
\IfFileExists{upquote.sty}{\usepackage{upquote}}{}
% use microtype if available
\IfFileExists{microtype.sty}{%
\usepackage[]{microtype}
\UseMicrotypeSet[protrusion]{basicmath} % disable protrusion for tt fonts
}{}
\IfFileExists{parskip.sty}{%
\usepackage{parskip}
}{% else
\setlength{\parindent}{0pt}
\setlength{\parskip}{6pt plus 2pt minus 1pt}
}
\usepackage{hyperref}
\hypersetup{
            pdftitle={MUltlog 1.14 \& iLC 1.2},
            pdfborder={0 0 0},
            breaklinks=true}
\urlstyle{same}  % don't use monospace font for urls
\setlength{\emergencystretch}{3em}  % prevent overfull lines
\providecommand{\tightlist}{%
  \setlength{\itemsep}{0pt}\setlength{\parskip}{0pt}}
\setcounter{secnumdepth}{5}
% Redefines (sub)paragraphs to behave more like sections
\ifx\paragraph\undefined\else
\let\oldparagraph\paragraph
\renewcommand{\paragraph}[1]{\oldparagraph{#1}\mbox{}}
\fi
\ifx\subparagraph\undefined\else
\let\oldsubparagraph\subparagraph
\renewcommand{\subparagraph}[1]{\oldsubparagraph{#1}\mbox{}}
\fi

% set default figure placement to htbp
\makeatletter
\def\fps@figure{htbp}
\makeatother


\title{MUltlog 1.14 \& iLC 1.2}
\date{}

\begin{document}
\maketitle

{
\setcounter{tocdepth}{3}
\tableofcontents
}
MUltlog is a system which takes as input the specification of a
finitely-valued first-order logic and produces a sequent calculus, a
natural deduction system, and clause formation rules for this logic. All
generated rules are optimized regarding their branching degree. The
output is in the form of a scientific paper written in LaTeX.

iLC is an editor for Tcl/Tk, which allows to specify many-valued logics
for MUltlog in a convenient form.

Further information is available on the
\href{http://www.logic.at/multlog/}{project webpage}, where you can also
find an up-to-date copy of this manual, and example outputs.

\hypertarget{requirements}{%
\section{Requirements}\label{requirements}}

You need the following to run MUltlog:

\begin{itemize}
\item
  MUltlog itself. The source code is available in the
  \href{https://github.com/rzach/multlog}{Multlog GitHub repository}.
\item
  Some standard Prolog system,
  e.g.~\href{https://www.swi-prolog.org/}{SWI-Prolog}. Other Prologs
  should work as well; MUltlog has been tested with SWI and SICStus
  Prolog.
\end{itemize}

The output of MUltlog is in the form of a LaTeX paper. To view it
properly, you need the typesetting system

\begin{itemize}
\tightlist
\item
  TeX, available from \href{https://ctan.org/}{CTAN}. For Linux, the
  \href{https://www.tug.org/texlive/}{TeXLive} distribution is
  particularly convenient to install and is most likely available via
  your package manager.
\end{itemize}

MUltlog includes a special editor, iLC, which allows to specify
many-valued logics in a convenient, windows-oriented way, instead of
typing an ASCII text in a strict syntax. To use this editor you need the
script language

\begin{itemize}
\tightlist
\item
  \href{https://www.tcl.tk/}{Tcl/Tk} (version 7.4/4.0 or later). Many
  Linux systems include Tcl/Tk by default (check for a program named
  \texttt{wish}).
\end{itemize}

\hypertarget{installation}{%
\section{Installation}\label{installation}}

\hypertarget{obtaining-multlog}{%
\subsection{Obtaining MUltlog}\label{obtaining-multlog}}

Get the newest release of MUltlog by cloning or downloading the Git
repository from
\href{https://github.com/rzach/multlog}{github.com/rzach/multlog}.

\hypertarget{running-the-installation-script}{%
\subsection{Running the installation
script}\label{running-the-installation-script}}

As of version 1.05, MUltlog comes with an installation script for Linux.
Before running the script:

\begin{itemize}
\item
  decide which Prolog to use. The script will look for SWI-Prolog,
  SICStus, and BinProlog in some standard locations, and suggest the
  result as a default to the user.
\item
  decide where to put MUltlog. If run as root, the default locations are
  \texttt{/usr/local/bin} for executables, \texttt{/usr/local/lib} for
  library files, and \texttt{/usr/share/doc} for documentation. If not
  run as root, the script will install into
  \texttt{\textasciitilde{}/.local/bin,}
  \texttt{\textasciitilde{}/.local/lib}, and
  \texttt{\textasciitilde{}/.local/doc}. Note that these directories
  must exist; the script will not try to create them.
\end{itemize}

To run the installation script, change to the installation directory
\texttt{multlog} and type

\begin{verbatim}
./ml_install
\end{verbatim}

The script will - determine the location of some Unix commands - ask the
user for the Prolog to use - ask the user for the place where to put
MUltlog - generate the deinstallation script \texttt{ml\_deinstall} -
insert the correct paths in some of MUltlog's files - copy the MUltlog
files in the right places.

In case of problems see the section on
\protect\hyperlink{troubleshooting}{troubleshooting} below.

Note that the installation procedure puts path information directly into
some of MUltlog's files. This means that to install MUltlog somewhere
else, you need the original distribution as well as the installation
script.

\hypertarget{deinstallation}{%
\subsection{Deinstallation}\label{deinstallation}}

Run the script

\begin{verbatim}
ml_deinstall
\end{verbatim}

to remove files installed by \texttt{ml\_install}. The deinstallation
script is located in the same directory as the other MUltlog commands
like \texttt{lgc2tex}, \texttt{lgc2pdf}, \ldots{}
(\texttt{/usr/local/bin} or \texttt{\textasciitilde{}/.local/bin} by
default).

\hypertarget{using-multlog}{%
\section{Using MUltlog}\label{using-multlog}}

The examples below assume that MUltlog was installed into the standard
place \texttt{/usr/local/*}, and assumes that the locations of the
MUltlog scripts \texttt{lgc2tex} and \texttt{lgc2pdf} are your command
path. If you use different settings, change the examples accordingly.

\hypertarget{guide-for-the-impatient}{%
\subsection{Guide for the impatient}\label{guide-for-the-impatient}}

\begin{itemize}
\item
  Move to a temporary directory, e.g.,

\begin{verbatim}
mkdir tmp; cd tmp
\end{verbatim}
\item
  Get the sample logic from the \texttt{doc} directory, e.g.,

\begin{verbatim}
cp /usr/share/doc/multlog/sample.lgc .
\end{verbatim}
\item
  Generate the paper in PDF format

\begin{verbatim}
lgc2pdf sample
\end{verbatim}

  You should now be able to open \texttt{goedel.pdf} using the PDF
  reader of your choice.
\item
  To edit the specification of the logic before generating the paper,
  type

\begin{verbatim}
ilc &
\end{verbatim}

  Select ``Open'' from the menu ``File'' and type \texttt{goedel} as the
  name of the file to be loaded.
\end{itemize}

The \texttt{examples/} directory of the distribution contains other
example specification and configuration files.

\hypertarget{creating-the-specification-of-a-logic}{%
\subsection{Creating the specification of a
logic}\label{creating-the-specification-of-a-logic}}

You can either use your favourite text editor, or the ``interactive
Logic Creator'' \texttt{ilc}.

In the first case specify your many-valued logic in the syntax described
in the sample specification \texttt{/usr/share/doc/multlog/sample.lgc}
and save the result as \texttt{\textless{}name\textgreater{}.lgc}.

To start \texttt{ilc}, type

\begin{verbatim}
ilc &
\end{verbatim}

A window pops up, and you are able to edit a new logic or re-edit an
already exisiting one, and to save the result in a textual format
suitable for MUltlog. Note that you have to store the logic as
\texttt{\textless{}name\textgreater{}}, the extension \texttt{.lgc}
being added automatically.

\hypertarget{creating-the-paper-pdf}{%
\subsection{Creating the paper (PDF)}\label{creating-the-paper-pdf}}

To obtain the paper corresponding to your logic, type

\begin{verbatim}
lgc2pdf <name>
\end{verbatim}

where \texttt{\textless{}name\textgreater{}} is the name under which you
saved your logic. This invokes MUltlog as well as PDFLaTeX and BibTeX
(or alternatively, if \texttt{pdflatex} was not found upon installation,
LaTeX, BibTeX, \texttt{dvips} and \texttt{ps2pdf}).

If \texttt{\textless{}name\textgreater{}.bib} exists, it should contain
a bibliography entry with key \texttt{ml}, which will be cited as the
source for the definition of the logic.

Additionally, all files are deleted except the specification of the
logic and the PDF file.

View the resulting paper with any PDF reader, e.g.

\begin{verbatim}
acroread goedel.pdf &
evince goedel.pdf &
...
\end{verbatim}

\hypertarget{creating-the-paper-latex}{%
\subsection{Creating the paper (LaTeX)}\label{creating-the-paper-latex}}

If you are interested in the LaTeX source of the paper, use
\texttt{lgc2tex} instead of \texttt{lgc2pdf}:

\begin{verbatim}
lgc2tex <name>
\end{verbatim}

This will invoke MUltlog, but does neither LaTeXing nor cleaning up. It
will produce two files: \texttt{\textless{}name\textgreater{}.tex} and
\texttt{\textless{}name\textgreater{}.sty}.
\texttt{\textless{}name\textgreater{}.tex} is a template LaTeX file
which loads \texttt{\textless{}name\textgreater{}.sty}. The latter
contains the difinitions specific to your logic.

The source will be \texttt{\textless{}name\textgreater{}.tex} and will
require \texttt{\textless{}name\textgreater{}.sty} and to be compiled.
\texttt{\textless{}name\textgreater{}.sty} contains the definitions
produced by MUltlog.

\hypertarget{creating-the-paper-dvi}{%
\subsection{Creating the paper (DVI)}\label{creating-the-paper-dvi}}

The command

\begin{verbatim}
lgc2dvi <name>
\end{verbatim}

where \texttt{\textless{}name\textgreater{}} is the name under which you
saved your logic, will produce a DVI file of the paper.

\hypertarget{specification-of-a-logic}{%
\section{Specification of a logic}\label{specification-of-a-logic}}

The directory \texttt{/usr/share/doc/multlog} (or whatever you chose)
contains a documented example of the configuration file format (as does
the \texttt{doc} subdirectory of the source distribution itself),
\texttt{sample.lgc}.

To specify a logic, your specification (\texttt{.lgc}) file has to
contain the following:

\hypertarget{the-name-of-the-logic-mandatory}{%
\subsection{The name of the logic
(mandatory)}\label{the-name-of-the-logic-mandatory}}

Here you specify the name of the logic to be used in the PDF.

Syntax:

\begin{verbatim}
logic "<logname>".
\end{verbatim}

where \texttt{\textless{}logname\textgreater{}} is a string described by
the regular expression RE1

({[}
\texttt{!\#\$\%\&\textquotesingle{}()*+,./0-9:;\textless{}=\textgreater{}?@A-Z{[}\textbackslash{}{]}\^{}\_\textasciigrave{}a-z\{\textbar{}\}\textasciitilde{}-}{]}
\textbar{} \texttt{""} )*

of up to 40 characters. In other words, the string may consist of any
printable ASCII character, where quotes (") have to be doubled.
\texttt{\textless{}logname\textgreater{}} may contain LaTeX code (e.g.,
for accented characters), where \texttt{\textbackslash{}} does
\emph{not} need to be escaped.

Example:

\begin{verbatim}
logic "G\""odel".
\end{verbatim}

\hypertarget{truth-values-mandatory}{%
\subsection{Truth values (mandatory)}\label{truth-values-mandatory}}

You have to list the truth values of the logic you are defining. The
order in which they are listed will be the order used for the
presentation of truth tables. Every truth value may appear only once.

Syntax:

\begin{verbatim}
truth_values { <v 1>, ..., <v n> }.
\end{verbatim}

Example:

\begin{verbatim}
truth_values {f,*,t}.
\end{verbatim}

where each of the truth values \texttt{\textless{}v\ 1\textgreater{}},
\ldots{}, \texttt{\textless{}v\ n\textgreater{}} (\(n \ge 2\)) is
described by the regular expression RE2

( {[}\texttt{a}-\texttt{z}{]}{[}\texttt{A-Za-z0-9\_}{]}* \textbar{}
{[}\texttt{-+*\^{}\textless{}\textgreater{}=\textasciitilde{}?@\#\$\&}{]}+
\textbar{} \texttt{0} \textbar{} {[}\texttt{1-9}{]}{[}\texttt{0-9}{]}* )

The truth values may consist of up to 10 characters. Unless you specify
how they should be typeset in the corresponding \texttt{.cfg} file, the
paper will use the names \texttt{\textless{}vn\textgreater{}} in italics
in the generated PDF.

\hypertarget{designated-truth-values-mandatory}{%
\subsection{Designated truth values
(mandatory)}\label{designated-truth-values-mandatory}}

The designated truth values are usually those representing ``true''. The
particular order of the values is of no significance as indicated by the
braces. Every truth value may appear at most once.

Syntax:

\begin{verbatim}
designated_truth_values { <v 1>, ..., <v n> }.
\end{verbatim}

Example:

\begin{verbatim}
designated_truth_values {t}.
\end{verbatim}

where each of the truth values \textless{}v 1\textgreater{}, \ldots{},
(n\textgreater{}0) is described by the regular expression RE2 above and
may consist of up to 10 characters.

The choice of designated truth values has no effect on the generated
rules. However, they make a difference to what sequent, initial tableau,
or initial clause set has to be used to give a proof of an entailment.

\hypertarget{orderings-of-truth-values-optional}{%
\subsection{Orderings of truth values
(optional)}\label{orderings-of-truth-values-optional}}

By specifying an ordering on truth values, you can declare an operator
or quantifier as being the ``inf'' (greatest lower bound) or ``sup''
(least upper bound) operator with respect to the ordering.

Syntax:

\begin{verbatim}
ordering(<ordname>, "<ordspec>").
\end{verbatim}

where \texttt{\textless{}ordname\textgreater{}} is defined according to
RE2 above and may consist of up to 10 characters.
\texttt{\textless{}ordspec\textgreater{}} is a string of up to 200
characters satisfying RE1 above. This string is either a single chain,
or a set of chains in \texttt{\{}\ldots{}\texttt{\}}. A chain, in turn,
is a seqence of either elements separated by \texttt{\textless{}}, where
each element is either a truth value (as defined by
\texttt{truth\_values}) or itself a set of chains.

In order to avoid ambiguities, spaces may be used to separate the
\texttt{\textless{}} sign from truth values (which may also contain the
character \texttt{\textless{}}).

The semantics of order specifications is as follows:

\begin{itemize}
\tightlist
\item
  Chains like
  ``\texttt{a\ \textless{}\ b\ \textless{}\ c\ \textless{}\ d\ \textless{}\ e}''
  are interpreted as an abbreviation for
  ``\texttt{a\ \textless{}\ b,\ b\ \textless{}\ c,\ c\ \textless{}\ d,\ d\ \textless{}\ e}''.
\item
  Independent chains are collected in sets:
  ``\texttt{\{a\ \textless{}\ b,\ c\ \textless{}\ d\ \textless{}e,\ ...\}}'';
\item
  Sets and chains can be nested. E.g.,
  ``\texttt{a\ \textless{}\ \{b,\ c\ \textless{}\ d\}\ \textless{}\ e}''
  is the same as
  ``\texttt{\{a\ \textless{}\ b,\ a\ \textless{}\ c,\ c\ \textless{}\ d,\ b\ \textless{}\ e,\ d\ \textless{}\ e\}}''.
\end{itemize}

Let \(R\) be the relation defined by this specification. The ordering
induced by \(R\) is the smallest reflexive, anti-symmetric and
transitive relation containing \(R\). Note that truth values with
different names are treated as being different from each other. Hence a
specification containing \texttt{a\textless{}b} and
\texttt{b\textless{}a} induces no ordering, since anti-symmetry would
imply \texttt{a}=\texttt{b}.

Example:

\begin{verbatim}
ordering(linear, "f < * < t").
\end{verbatim}

\hypertarget{definitions-of-operators}{%
\subsection{Definitions of operators}\label{definitions-of-operators}}

Optional; but what's a logic without operators?

\hypertarget{mappings}{%
\subsubsection{Mappings}\label{mappings}}

In its simplest and most general form, each operator is specified by its
name and the mapping of input to output values. The braces indicate that
the order in which the input tuples are assigned output values is of no
significance. The definitions should be complete: every \(k\)-tuple has
to be assigned exactly one value, where \(k>0\) is the arity of the
operator. There may be several operators with the same
\texttt{\textless{}opname\textgreater{}} but with different
\texttt{\textless{}arity\textgreater{}}.

Syntax:

\begin{verbatim}
operator(<opname>/<arity>, mapping { <ass 1>, ..., <ass m> }).
\end{verbatim}

\texttt{\textless{}opname\textgreater{}} is defined according to RE2
above and may consist of up to 10 characters.
\texttt{\textless{}arity\textgreater{}} is a non-negative integer.
\texttt{\textless{}ass\ 1\textgreater{}}, \ldots{},
\texttt{\textless{}ass\ m\textgreater{}} are assignments of the form

\begin{verbatim}
( <v 1>, ..., <v k> ) : <v>
\end{verbatim}

where
\texttt{\textless{}v\ 1\textgreater{}},\ldots{},\texttt{\textless{}v\ k\textgreater{}},
and \texttt{\textless{}v\textgreater{}} are truth values. Assignments
must be separated by commas. For \(k = 0\), the mapping consists of a
single truth value.

Example:

\begin{verbatim}
operator(true /0, mapping { t }).
operator(and  /2, mapping { (t,t): t,
                            (t,*): *,
                            (t,f): f,
                            (*,t): *,
                            (*,*): *,
                            (*,f): f,
                            (f,t): f,
                            (f,*): f,
                            (f,f): f
                          }
\end{verbatim}

\hypertarget{tables}{%
\subsubsection{Tables}\label{tables}}

Binary operators can be also specified as tables. Since the order of
truth values in the table is significant, brackets are used instead of
braces.

Syntax:

\begin{verbatim}
operator(<opname>/2, table [ <v 1>, ..., <v m> ]).
\end{verbatim}

\texttt{\textless{}opname\textgreater{}} is defined according to RE2
above and may consist of up to 10 characters.
\texttt{\textless{}v\ 1\textgreater{}},\ldots{},\texttt{\textless{}v\ m\textgreater{}}
are truth values. The number of elements in the table, \(m\), has to be
equal to \((n+1)^2-1\), where \(n\) is the number of different truth
values.

Example:

\begin{verbatim}
operator(and/2, table   [     t, *, f,
                          t,  t, *, f,
                          *,  *, *, f,
                          f,  f, f, f
                        ]
        ).
\end{verbatim}

\hypertarget{inf-and-sup}{%
\subsubsection{Inf and sup}\label{inf-and-sup}}

Operators may also be declared to be the infimum (greatest lower bound)
or supremum (least upper bound) with respect to some user-defined
ordering.

Syntax:

\begin{verbatim}
operator(<opname>/<arity>, sup(<ordname>)).
operator(<opname>/<arity>, inf(<ordname>)).
\end{verbatim}

Here, \texttt{\textless{}opname\textgreater{}} is defined according to
RE2 above and may consist of up to 10 characters.
\texttt{\textless{}arity\textgreater{}} is a non-negative integer
greater than one. \texttt{\textless{}ordname\textgreater{}} is the name
of an appropriate ordering defined by an \texttt{ordering} statement.

``sup'' stands for the least-upper-bound (= supremum) operation w.r.t.
the given ordering. The value of the operator is determined as the least
upper bound of the input truth values in the ordering. ``inf'' stands
for the greatest-lower-bound (= infimum) operation w.r.t. the given
ordering. The value of the operator is determined as the greatest lower
bound of the input truth values in the ordering.

The ordering has to define a unique supremum/infimum for any two truth
values.

Examples:

\begin{verbatim}
operator(and  /2, inf(linear)).
operator(or   /2, sup(linear)).
\end{verbatim}

\hypertarget{definitions-of-distribution-quantifiers}{%
\subsection{Definitions of (distribution)
quantifiers}\label{definitions-of-distribution-quantifiers}}

Optional.

\hypertarget{mappings-1}{%
\subsubsection{Mappings}\label{mappings-1}}

In its simplest and most general form, each quantifier is specified by
its name and a mapping assigning a truth value to each non-empty subset
of the truth values. The definitions should be complete: \emph{every}
non-empty subset should be assigned exactly one value.

Syntax:

\begin{verbatim}
quantifier(<quname>, mapping { <ass 1>, ..., <ass m> }).
\end{verbatim}

\texttt{\textless{}quname\textgreater{}} is defined according to RE2
above and may consist of up to 10 characters.
\texttt{\textless{}ass\ 1\textgreater{}},\ldots{},\texttt{\textless{}ass\ m\textgreater{}}
are assignments of the form
\texttt{\{\ \textless{}v\ 1\textgreater{},\ ...,\ \textless{}v\ k\textgreater{}\ \}\ :\ \textless{}v\textgreater{}},
where
\texttt{\textless{}v\ 1\textgreater{}},\ldots{},\texttt{\textless{}v\ k\textgreater{}},
and \texttt{\textless{}v\textgreater{}} are truth values. \(k\) has to
be greater than one.

Example:

\begin{verbatim}
quantifier(forall, mapping { {t}    : t,
                             {t,*}  : *,
                             {t,f}  : f,
                             {t,*,f}: f,
                             {*}    : *,
                             {*,f}  : f,
                             {f}    : f
                           }
          ).
\end{verbatim}

\hypertarget{induced-quantifiers}{%
\subsubsection{Induced quantifiers}\label{induced-quantifiers}}

The definition of induced quantifiers in a more comfortable and less
error-prone form. Quantifiers can only be induced by operators which are
associative, commutative and idempotent.

Syntax:

\begin{verbatim}
quantifier(<quname>, induced_by <opname>/<arity>).
\end{verbatim}

\texttt{\textless{}quname\textgreater{}} is defined according to RE2
above and may consist of up to 10 characters.
\texttt{\textless{}opname\textgreater{}} is defined according to RE2
above and may consist of up to 10 characters, and should be an operator
defined as above. \texttt{\textless{}arity\textgreater{}} is an integer
greater than one.

Example:

\begin{verbatim}
quantifier(forall, induced_by and/2).
\end{verbatim}

\hypertarget{inf-and-sup-1}{%
\subsubsection{Inf and sup}\label{inf-and-sup-1}}

Quantifiers can also be induced by a lub/glb operator.

Syntax:

\begin{verbatim}
quantifier(<quname>, induced_by <bop>(<ordname>)).
\end{verbatim}

\texttt{\textless{}quname\textgreater{}} is defined according to RE2
above and may consist of up to 10 characters.
\texttt{\textless{}bop\textgreater{}} is either sup or inf.
\texttt{\textless{}ordname\textgreater{}} is the name of an appropriate
ordering defined by an ``\texttt{ordering}''-statement.

Example:

\begin{verbatim}
quantifier(forall, induced_by inf(linear)).
\end{verbatim}

\hypertarget{interactive-logic-creator}{%
\subsection{interactive Logic Creator}\label{interactive-logic-creator}}

\texttt{ilc} is a graphical front-end to make creating \texttt{.lgc}
files a little easier. It requires Tcl/Tk (version 7.4/4.0 or later). To
be exact, you only need the executable \texttt{wish} (the windowing
shell) but not the libraries and none of the extensions.

The program should be rather self-explanatory once you know what goes
into the \texttt{.lgc} file. You can load and save \texttt{.lgc} files
from the ``File'' menu. Before you can specify orderings, operators, and
quantifiers, you have to enter the name of the logic and the list of
truth values.

\hypertarget{tex-configuration-files}{%
\section{TeX configuration files}\label{tex-configuration-files}}

MUltlog will generate a \texttt{.tex} and \texttt{.sty} file from a
given \texttt{.lgc} file. If present, it will also use the content of a
corresponding \texttt{.cfg} file, which can be used to fine-tune the
formatted output.

The \texttt{.cfg} file for a logic can contain three kinds of
declaration:

\begin{itemize}
\item
  \texttt{texName(\textless{}name\textgreater{},\ \textless{}definition\textgreater{})}
  associates the name \texttt{\textless{}name\textgreater{}} of a truth
  value, operator, or quantifier used in the \texttt{.lgc} file with
  LaTeX code used to typeset it. For instance, if \texttt{t} is a truth
  value in the \texttt{.lgc} file, then

\begin{verbatim}
texName(t, "\\mathbf{T}").
\end{verbatim}

  will result in the truth value \texttt{t} be typset as \(\mathbf{T}\).
  (Note that \texttt{\textbackslash{}} have to be doubled.) If the LaTeX
  replacement is a simple command beginning with a lowercase letter, the
  quotation marks can be left off, e.g.,

\begin{verbatim}
texName(and,    \\wedge  ).
\end{verbatim}
\item
  \texttt{texInfix(\textless{}op\textgreater{})} and
  \texttt{texPrefix(\textless{}op\textgreater{})} will cause formulas
  involving an operator to be set either as infix or prefix, as opposed
  to the default operator notation. Of course, \texttt{texInfix} can
  only be applied to binary operators, and \texttt{texPrefix} only to
  unary operators. For instance:

\begin{verbatim}
texInfix(or).
texPrefix(neg).
\end{verbatim}
\item
  \texttt{texExtra(\textless{}macro\textgreater{},\ \textless{}definition\textgreater{})}
  will insert a definition for
  \texttt{\textbackslash{}\textless{}macro\textgreater{}} with body
  \texttt{\textless{}definition\textgreater{}} in the \texttt{.sty}
  file, which will be loaded in the preamble. This can be used to
  (re-)define any macro used in the LaTeX file. For instance,

\begin{verbatim}
texExtra("ShortName", "\\textbf{\\L}_3").
\end{verbatim}

  will define \texttt{\textbackslash{}ShortName} as
  \texttt{\textbackslash{}textbf\{\textbackslash{}L\}\_3}, and this
  macro will be available in the LaTeX file. The LaTeX template file
  makes use of a number of macros which can be defined in this way:

  \begin{itemize}
  \item
    \texttt{\textbackslash{}Preamble} will be executed in the preamble
    just before \texttt{\textbackslash{}begin\{document\}}. It can be
    used, e.g., to load packages needed for some of the operator
    symbols, or to change the document font.
  \item
    \texttt{\textbackslash{}ShortName} may contain code for an
    abbreviation or symbol for the logic.
    \texttt{\textbackslash{}ShortName} will be called in math mode.
  \item
    \texttt{\textbackslash{}FullNameOfLogic} is the macro used to insert
    the name of the logic. By default it will be
    ``\texttt{\textless{}logname\textgreater{}} logic''. Sometimes this
    doesn't work, so, e.g., you could say:

\begin{verbatim}
texExtra("FullNameOfLogic", "Halld\\'en's logic of nonsense").
\end{verbatim}
  \item
    \texttt{\textbackslash{}Intro} will be called (if defined) after the
    first paragraph of the introduction, and can be used to print a
    paragraph on the history or motivation of the logic. This paragraph
    can use \texttt{\textbackslash{}cite} to generate references to any
    entries in \texttt{\textless{}logname\textgreater{}.bib}.
  \item
    \texttt{\textbackslash{}Semantics} will be called just before the
    definition of the matrix of the logic. It can be used to print a
    paragraph, say, about the intuitive interpretation of the truth
    values, or how the truth functions of the operators are defined
    (say, on the basis of an ordering).
  \item
    \texttt{\textbackslash{}Link} will be added as a download link to
    the citation information at the bottom of the first page.
  \end{itemize}
\end{itemize}

\hypertarget{trouble-shooting}{%
\section{Trouble shooting}\label{trouble-shooting}}

\hypertarget{installation-errors}{%
\subsection{Installation errors}\label{installation-errors}}

The installation script may produce the following warnings and errors.

\begin{itemize}
\item
  ``Error: \texttt{\textless{}directory\textgreater{}} does not exist.''

  The installation script did not find the directory for executables,
  library, or documentation (\texttt{/usr/local/bin},
  \texttt{/usr/local/lib}, and \texttt{/usr/share/doc} or
  \texttt{\textasciitilde{}/.local/bin},
  \texttt{\textasciitilde{}/.local/lib},
  \texttt{\textasciitilde{}./local/doc} by default). Create the
  directories before running the script or select different directories
  when prompted.

  \begin{itemize}
  \tightlist
  \item
    ``Error: could not find Unix command
    \texttt{\textless{}command\textgreater{}}.'' where
    \texttt{\textless{}command\textgreater{}} is one of
  \end{itemize}

\begin{verbatim}
basename chmod cp dirname false grep mkdir pwd rm sed true.
\end{verbatim}

  The installation script and the scripts for starting MUltlog
  (\texttt{lgc2tex}, \texttt{lgc2dvi}, \texttt{lgc2pdf}, and
  \texttt{lgc2ilc}) need these Unix commands. The error message means
  that could not be located, neither on the current command search path
  nor in the directories \texttt{/usr/local/bin},
  \texttt{/usr/local/sbin}, \texttt{/usr/bin}, \texttt{/usr/sbin},
  \texttt{/bin}, or \texttt{/sbin}. Locate the directory containing
  \texttt{\textless{}command\textgreater{}} and put it on your command
  search path during installation.

  If your Unix system does not have
  \texttt{\textless{}command\textgreater{}} at all, submit an issue on
  https://github.com/rzach/multlog/.
\item
  ``Warning: could not find TeX command
  \texttt{\textless{}command\textgreater{}}.'' where
  \texttt{\textless{}command\textgreater{}} is one of

\begin{verbatim}
latex bibtex.
\end{verbatim}

  The script \texttt{lgc2dvi} needs latex and bibtex to produce a
  DVI-file from the TeX document created by MUltlog. The warning means
  that \texttt{\textless{}command\textgreater{}} could not be located,
  neither on the current command search path nor in the directories

  \texttt{/usr/local/bin} \texttt{/usr/local/}sbin
  /usr/bin\texttt{}/usr/sbin\texttt{}/bin\texttt{}/sbin`

  Check whether TeX is properly installed and put the directory
  containing \texttt{\textless{}command\textgreater{}} on your command
  search path during installation.
\item
  ``Warning: couldn't find any PDF converters.''

  The script \texttt{lgc2pdf} needs either \texttt{pdflatex} or
  \texttt{latex}, \texttt{dvips} and \texttt{ps2pdf} to produce a
  PDF-file from the TeX document created by MUltlog. The warning means
  that either \texttt{pdflatex} or \texttt{dvips} and \texttt{ps2pdf}
  could not be located, neither on the current command search path nor
  in the directories \texttt{/usr/local/bin}, \texttt{/usr/local/sbin},
  \texttt{/usr/bin}, \texttt{/usr/sbin}, \texttt{/bin}, or
  \texttt{/sbin}. Check whether TeX and Ghostscript are properly
  installed and put the directory containing the PDF converter on your
  command search path during installation.
\item
  ``Warning: could not find Tcl/Tk command wish.''

  The editor \texttt{ilc} needs the Tcl/Tk package, in particular the
  program \texttt{wish}. The warning means that \texttt{wish} could not
  be located, neither on the current command search path nor in the in
  the directories \texttt{/usr/local/bin}, \texttt{/usr/local/sbin},
  \texttt{/usr/bin}, \texttt{/usr/sbin}, \texttt{/bin}, or
  \texttt{/sbin}. Check whether Tcl/Tk is properly installed and put the
  directory containing \texttt{wish} on your command search path during
  installation.
\item
  ``Error: \texttt{\textless{}command\textgreater{}} does not exist or
  has no execute permission.''

  \texttt{\textless{}command\textgreater{}} (suggested by the user as
  Prolog interpreter) does not exist or cannot be executed.
\item
  ``Error: \texttt{\textless{}command\textgreater{}} does not behave
  like Prolog.''

  \texttt{\textless{}command\textgreater{}} (suggested by the user as
  Prolog interpreter) exists but fails the test performed by the
  installation script. This test is a heuristic check whether
  \texttt{\textless{}command\textgreater{}} is indeed a Prolog system;
  more precisely, the output of

\begin{verbatim}
echo 'f(X,not)=f(ger,Y), print(X),print(Y),halt.' | <command>
\end{verbatim}

  is checked for the string ``gernot''. If
  \texttt{\textless{}command\textgreater{}} is a Prolog system but fails
  this test, or if it is no Prolog system but passes the test, submit an
  issue on https://github.com/rzach/multlog/.
\end{itemize}

\hypertarget{runtime-errors}{%
\subsection{Runtime errors}\label{runtime-errors}}

\begin{itemize}
\item
  "warning: *** copy\_term overflow in findall\_store\_heap ***"
\item
  "warning: *** overflow in findall\_store\_heap ***"

  These error messages are produced by BinProlog and indicate that the
  heap size is too small, which by default is 512 kB. (Reported by
  Matthew Spinks
  \href{mailto:mspinks@mugc.cc.monash.edu.au}{\nolinkurl{mspinks@mugc.cc.monash.edu.au}}.)

  Solutions (alternatives):

  \begin{itemize}
  \item
    Switch to SWI Prolog which is faster and more robust. You will have
    to re-install Multlog after SWI-Prolog is in its place.
  \item
    Re-install MUltlog using BinProlog with the option \texttt{-h\ 1024}
    (or some bigger value). When the installation script asks for the
    Prolog to be used, type e.g.~\texttt{/usr/local/bin/bp\ -h\ 1024}.
    Note that adding options to the Prolog call only works with MUltlog
    1.06 or higher.
  \item
    Add the option \texttt{-h\ 1024} by hand. To this aim, edit the
    files

\begin{verbatim}
/usr/local/bin/lgc2tex
/usr/local/bin/lgc2dvi
/usr/local/bin/lgc2pdf
/usr/local/lib/multlog/ilc/lgc2ilc
\end{verbatim}

    Near their top there is a line starting with \texttt{PROLOG=}.
    Replace this line e.g.~by

\begin{verbatim}
PROLOG='/usr/local/bin/bp -h 1024'
\end{verbatim}

    Make sure that the files have still execute permission after saving.
  \end{itemize}
\end{itemize}

\hypertarget{about-multlog}{%
\section{About MUltlog}\label{about-multlog}}

The following people contributed to MUltlog (in alphabetical order):

\begin{itemize}
\tightlist
\item
  Stefan Katzenbeisser and Stefan Kral rewrote the optimization
  procedure for operators using more efficient data structures
\item
  Andreas Leitgeb is the author of iLC, the interactive Logic Creator
\item
  Wolfram Nix wrote eLK, a DOS interface to MUltlog
\item
  Alexandra Pascal wrote JMUltlog, a JAVA interface to MUltlog
\item
  Gernot Salzer wrote the MUltlog kernel and coordinated the project.
\item
  Markus Schranz wrote a web interface to MUltlog based on plain HTML
  and Perl.
\end{itemize}

(eLK, JMUltlog, and the web interface are no longer available.)

\end{document}
